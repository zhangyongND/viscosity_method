% Thank Bogdan for this template :-)

\documentclass[12pt]{article}
%\documentclass[12pt,prb,preprint]{revtex4}

\setlength{\textwidth}{6.5in} % sets width of text to 6.5 in
\renewcommand{\baselinestretch}{1.5} % double space; {1} for single space
\textwidth 6.5in % default width = 5.5 in; def margins 1.5 in
\setlength{\evensidemargin}{0.0 cm}
\setlength{\oddsidemargin}{0.0 cm} % {1.0 in} shifts text right 1 in

\textheight 8.5 in % sets height of text to 8.5 in
\topmargin -0.0 in % sets top margin 0.1 in above def (2 in)


%------------------
\title{Reliable Viscosity from Equilibrium Molecular Dynamics Simulations: A Time Decomposition Method}
\author{Yong Zhang \ and Edward J. Maginn\thanks{Corresponding author. E-mail: ed@nd.edu} \\
                Department of Chemical and Biomolecular Engineering\\
                University of Notre Dame\\
                Notre Dame, Indiana, 46556 USA}
%\renewcommand{\thefootnote}{\fnsymbol{footnote}}
%-----------------


\date{\today}
\usepackage[dvips]{graphicx}
%\usepackage{graphicx}
\usepackage{overcite}
\usepackage{longtable}
\usepackage{supertabular}

\begin{document}
\pagenumbering{arabic}
%\baselineskip 0.9 cm

\maketitle

%%%%%%%%%%%%%%%%%%%%%%%%%%%%%%%%%%%%%%%%%%%%%%%%%%%%%%%%%%%%%%%%%%%%%%%
\newpage
\begin{abstract}

%\newline

{\bf Keywords:} 

\end{abstract}

%%%%%%%%%%%%%%%%%%%%%%%%%%%%%%%%%%%%%%%%%%%%%%%%%%%%%%%%%%%%%%%%%%%%%%%
\newpage
\section{Introduction}



%%%%%%%%%%%%%%%%%%%%%%%%%%%%%%%%%%%%%%%%%%%%%%
\section{Simulation Procedure}

%%%%%%%%%%%%%%%%%%%%%%%%%%%%%%%%%%%%%%%%%%%%%%
\section{Results and Discussions}

{\bf Notes:

FOR CORRELATION TIME CUTOFF: There are two reasons to fit the correlation function at as short as possible.
One reason is, as shown in the calculated $\sigma$, the standard error increases with the increasing correlation time.Secondly, computationally, it is always prefer to run the simulations shorter as long as the results are reliable.

Experimental deviates.
There are only four points from simulation, which 

The goal of the current work is not to calculate the viscosities for one or two examples,
but rather to describe a general method for the calculation of viscosity reliablly.
However, for the two examples studied in this work,
the calculated viscosity using the proposed procedure
do agree with available experimental results reasonable well
although only the general Amber force field was used,
which was not optimized for the such calculation.
If a dedicated force field is used,
the results are likely further improved.}




%%%%%%%%%%%%%%%%%%%%%%%%%%%%%%%%%%%%%%%%%%%%%%
\section{Concluding Remarks}

%%%%%%%%%%%%%%%%%%%%%%%%%%%%%%%%%%%%%%%%%%%%%%
\vspace{7 mm}
{\bf\Large Supporting Information}


\vspace{7 mm}
{\bf\Large Acknowledgment}

This material is based upon work supported by
the National Science Foundation (NSF) Partnerships for Innovation (PFI) subprogram:
Building Innovation Capacity (BIC), award number 1237829.
Computational resources were provided by the Center for Research Computing (CRC) at the University of Notre Dame.
{\bf JECSR NERSC}

%%%%%%%%%%%%%%%%%%%%%%%%%%%%%%%%%%%%%%%%%%%%%%%%%%%%%%%%%%%%%%%%%%%%%%%%%%

%--------ref---------------
\newpage
\clearpage

\bibliographystyle{jpc_new}
%\bibliographystyle{proteins}
%\bibliographystyle{achemso}
%\bibliographystyle{jcp}
\bibliography{paper}


%%%%%%%%%%%%%%%%%%%%%%%%%%% tables


%%%%%%%%%%%%%%%%%%%%%%%%%%% figures

\newpage
\clearpage
{\bf\Large Figure captions}
\vspace{7 mm}


%%%%%%%%%%%%%%%%%%%%%%%%%%%
\newpage
\clearpage
\begin{figure}
\begin{center}
\includegraphics[angle=0,width=6.0in]{/pscratch/yzhang19/20-viscosity/15-ethanol/03-nvt/visc-2.eps}
\caption{Ethanol}
\label{fig:visc-longtraj}
\end{center}
\end{figure}



\newpage
\clearpage
\begin{figure}
\begin{center}
\includegraphics[angle=0,width=6.0in]{/pscratch/yzhang19/20-viscosity/15-ethanol/04-nvt-swarm/298/ave/viscosity-3.eps}
\caption{Ethanol}
\label{fig:visc-multi}
\end{center}
\end{figure}

\newpage
\clearpage
\begin{figure}
\begin{center}
\includegraphics[angle=0,width=3.2in]{/pscratch/yzhang19/20-viscosity/15-ethanol/04-nvt-swarm/visc-2.eps}
\vspace {0.2in}
\includegraphics[angle=0,width=3.2in]{/pscratch/yzhang19/20-viscosity/15-ethanol/04-nvt-swarm/visc-1.eps}
\\
\includegraphics[angle=0,width=3.2in]{/pscratch/yzhang19/20-viscosity/15-ethanol/04-nvt-swarm/visc-3.eps}
\vspace {0.2in}
\includegraphics[angle=0,width=3.2in]{/pscratch/yzhang19/20-viscosity/15-ethanol/04-nvt-swarm/visc-22.eps}
\caption{Cutoff for fit.}
\label{fig:cutoff-ethanol}
\end{center}
\end{figure}

\newpage
\clearpage
\begin{figure}
\begin{center}
\includegraphics[angle=0,width=3.2in]{/pscratch/yzhang19/20-viscosity/15-ethanol/04-nvt-swarm/298/visc-conv-1.eps}
\\
\includegraphics[angle=0,width=3.2in]{/pscratch/yzhang19/20-viscosity/02-C4C1im/04-nvt-swarm/350/visc-conv-1.eps}
\caption{Viscosity convergence.}
\label{fig:visc-conv}
\end{center}
\end{figure}


\newpage
\clearpage
\begin{figure}
\begin{center}
\includegraphics[angle=0,width=3.2in]{/pscratch/yzhang19/20-viscosity/post_analysis_BmimTf2n_Ethanol/density-2.eps}
\caption{Ethanol}
\label{fig:density-ethanol}
\end{center}
\end{figure}


\newpage
\clearpage
\begin{figure}
\begin{center}
\includegraphics[angle=0,width=3.2in]{/pscratch/yzhang19/20-viscosity/post_analysis_BmimTf2n_Ethanol/visc-ethanol-1.eps}
\\
\includegraphics[angle=0,width=3.2in]{/pscratch/yzhang19/20-viscosity/post_analysis_BmimTf2n_Ethanol/visc-ethanol-2.eps}
\caption{Ethanol}
\label{fig:visc-ethanol}
\end{center}
\end{figure}


\newpage
\clearpage
\begin{figure}
\begin{center}
\includegraphics[angle=0,width=3.2in]{/pscratch/yzhang19/20-viscosity/post_analysis_BmimTf2n_Ethanol/density-1.eps}
\caption{Bmim-TF2N}
\label{fig:density-bmim}
\end{center}
\end{figure}


\newpage
\clearpage
\begin{figure}
\begin{center}
\includegraphics[angle=0,width=3.2in]{/pscratch/yzhang19/20-viscosity/post_analysis_BmimTf2n_Ethanol/visc-bmim-2.eps}
\\
\includegraphics[angle=0,width=3.2in]{/pscratch/yzhang19/20-viscosity/post_analysis_BmimTf2n_Ethanol/visc-bmim-1.eps}
\caption{Top: BMIM-TF2N viscosity extrapolation from simulation data.
Bottom: BMIM-TF2N viscosity extrapolation from experimental data.}
\label{fig:visc-bmim}
\end{center}
\end{figure}
%%%%%%%%%%%%%%%%%%%%%%%%%%%



\end{document}
